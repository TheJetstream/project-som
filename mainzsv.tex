\documentclass{beamer}
\usetheme{Goettingen}

\usepackage[german]{babel}
\usepackage[utf8]{inputenc}
\usepackage{hyperref}
\usepackage{graphicx}

\title{Buchempfehlungen mit Hilfe von Selbstorganisierende Karten}
\subtitle{Zwischenstandsverteidigung}
\author{Julius Quasebarth \and Luisa Derer \and Robin Hankel}
\institute{Albert Schweitzer Gymnasium Erfurt, Spez.}
\date{16. März 2015}

\begin{document}

\maketitle

\begin{frame}{Gliederung}
\tableofcontents
\end{frame}

\section{Bisher Erreichtes}

\subsection{Informatischer Anteil}

\begin{frame}{Programm}
\begin{block}{Funktionalität}
Umsetzung von...
\pause
\begin{itemize}[<+->]
\item grundlegend benötigten Datenstrukturen
\item Training selbstorganisierender Karten
\item diversen Hilfsfunktionen
\end{itemize}
\end{block}
\pause
\begin{block}{Dokumentation}
\begin{itemize}[<+->]
\item Dokumentation vorhanden
\item einsehbar auf \url{http://sammex.github.io/project-somc}
\item Projekt mit Aufgabenverwaltung einsehbar auf \url{http://github.com/sammex/project-somc}
\end{itemize}
\end{block}
\end{frame}

\begin{frame}{Über Haskell}
Haskell ist \emph{funktional}.
\begin{itemize}[<+->]
\item kein Wert darf geändert werden (näher an der Mathematik: \(x = x +  3\) ist ungültig)
	\begin{itemize}[<+->]
	\item Funktionen mit gleichen Argumenten geben gleiche Ausgaben zurück
	\item IO muss durch \glqq{}Monads\grqq{} verarbeitet werden
	\item nicht-imperativer Stil (alles ist ein großer Ausdruck)
	\item es existieren nur Funktionen (Variablen = Funktionen ohne Argumente)
	\end{itemize}
\end{itemize}
\end{frame}

\subsection{Schriftlicher Anteil}

\begin{frame}{Kapitel}
\begin{itemize}[<+->]
\item Einleitung fertiggestellt
\item englisches \glqq{}Abstract\grqq{} nicht geschrieben
\item Kapitel 2 \& 3 geschrieben (beschreiben Grundlagen)
\end{itemize}
\end{frame}

\section{Vergleich zum Zeitplan}

\begin{frame}{Vergleich zum Zeitplan}
\begin{itemize}[<+->]
\item Lernen von Haskell länger als geplant
\item Schreiben des Programmes verbraucht mehr Zeit als geplant
\item Schreiben einzelner Kapitel schneller
\end{itemize}
\end{frame}

\section{Präsentation des Erarbeiteten}

\begin{frame}{Programm, API}
\begin{itemize}
\item \url{http://sammex.github.io/project-somc}
\end{itemize}
\end{frame}

\end{document}