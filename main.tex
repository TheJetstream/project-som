\documentclass[a4paper]{article}

\usepackage[english,ngerman]{babel}
\usepackage[utf8]{inputenc}
\usepackage{amsmath}
\usepackage{graphicx}
\usepackage[colorinlistoftodos]{todonotes}

\title{Buchempfehlungen mit Hilfe von selbstorganisierenden Karten erstellen}

\author{Julius Quasebarth \and Luisa Derer \and Robin Hankel \\ Betreuer: Johannes Suepke}

\date{\today}

\begin{document}
\maketitle

\begin{otherlanguage}{english}
\begin{abstract}
Our super duper cool abstract text...
\end{abstract}
\end{otherlanguage}

\tableofcontents

\section{Einleitung}

Jeder kennt das Gefühl der Leere nach dem Lesen eines Buches. Welches Buch soll man sich als nächstes vornehmen? In dieser Seminarfacharbeit lösen wir dieses Problem: Es geht um Buchempfehlungen mit Hilfe von selbst organisierenden Karten. Mit diesem mathematischen Prinzip lassen sich Punkt im n-dimensionalem Raum in beliebig viele Gruppen unterteilen, wodurch sich Gruppen von Personen mit ähnlichen Lesevorzügen erstellen lassen. Je nachdem, welche Bücher die Personen bereits gelesen haben, kann somit ein \glqq{}Pool\grqq{} aus Büchern erstellt werden. In unserer Seminarfacharbeit wird vor allem auf die Erstellung des Programms, das Sammeln von Daten dafür und die Anwendung dessen eingegangen.

\section{Mathematische Prinzipien}

\section{Anwendung dieser Prinzipien}

\section{Erklärung / Vorstellung des Programms}

\subsection{Funktionsweise des abstrakten Teils}

\subsection{Erweiterung auf unsere Anwendung / Anpassung des Programms}

\subsection{Anwendung des Programms}

\end{document}