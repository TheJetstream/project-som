%%%%%%%%%%%%%%%%%%%%%%%%%%%%%%%%%%%%%%%%%
% Journal Article
% LaTeX Template
% * <feldrandstudios@gmail.com> 2015-01-30T20:40:28.601Z:
%
% 
%
% Version 1.3 (9/9/13)
%
% This template has been downloaded from:
% http://www.LaTeXTemplates.com
%
% Original author:
% Frits Wenneker (http://www.howtotex.com)
%
% License:
% CC BY-NC-SA 3.0 (http://creativecommons.org/licenses/by-nc-sa/3.0/)
%
%%%%%%%%%%%%%%%%%%%%%%%%%%%%%%%%%%%%%%%%%

\documentclass[twoside,a4paper,draft]{article}

% --- DOC CONFIG ---

\usepackage[sc]{mathpazo}
\usepackage[T1]{fontenc}
\linespread{1.05}
\usepackage[protrusion=true,expansion=true,final]{microtype}

\usepackage[hmarginratio=1:1,top=32mm,columnsep=20pt]{geometry}
\usepackage{multicol}
\usepackage[hang, small,labelfont=bf,up,textfont=it,up]{caption}
\usepackage{booktabs}
\usepackage{float}
\usepackage{hyperref}

\usepackage{lettrine}
\usepackage{paralist}

\usepackage{abstract}

\usepackage{titlesec}
\titleformat{\section}[block]{\large\scshape\centering}{\thesection.}{1em}{}
\titleformat{\subsection}[block]{\large}{\thesubsection.}{1em}{}

\usepackage{fancyhdr}
\pagestyle{fancy}
\fancyhead{}
\fancyfoot{}
\fancyhead[C]{Selbstorganisierende Karten \(\bullet\) 2014 / 2015 \(\bullet\) ASG Spez.}
\fancyfoot[RO,LE]{\thepage}

\usepackage[english,ngerman]{babel}
\usepackage[utf8]{inputenc}
\usepackage{amsmath,amsthm,amsfonts}
\usepackage{graphicx}
\usepackage[colorinlistoftodos]{todonotes}

% --- LITTLE TWEAKS ---

\renewcommand{\abstractnamefont}{\normalfont\bfseries}
\renewcommand{\abstracttextfont}{\normalfont\small\itshape}

\renewcommand{\thesection}{\Roman{section}}
\renewcommand{\thesubsection}{\arabic{subsection}}

\newcommand{\commonlettrine}[1]{\lettrine[nindent=0em,lines=2]{#1}}

% --- DOCUMENT INFORMATION ---

\title{\vspace{-15mm}\fontsize{24pt}{10pt}\selectfont\bfseries{}Buchempfehlungen mit Hilfe von selbstorganisierenden Karten erstellen}

\author{\large\textsc{Julius Quasebarth \quad Luisa Derer \quad Robin Hankel}\thanks{Projektbetreuer: Johannes Suepke, Außenbetreuer: ???}\\[2mm]\normalsize Alber Schweitzer Gymnasium Erfurt (Spez.)\\\vspace{-5mm}}

\date{2014 / 2015}

% --- BEGIN OF DOCUMENT ---

\begin{document}

\maketitle

\thispagestyle{fancy}

% --- BEGIN OF TEXT ---

\begin{otherlanguage}{english}
\begin{abstract}
\noindent
Our super duper cool abstract text...
\end{abstract}
\end{otherlanguage}

\tableofcontents

\begin{multicols}{2}

\section{Einleitung}

\commonlettrine{J}eder kennt das Gefühl der Leere nach dem Lesen eines Buches. Welches Buch soll man sich als nächstes vornehmen? In dieser Seminarfacharbeit lösen wir dieses Problem: Es geht um Buchempfehlungen mit Hilfe von selbstorganisierenden Karten. Mit diesem mathematischen Prinzip lassen sich Punkt im n-dimensionalem Raum in beliebig viele Gruppen unterteilen, wodurch sich Gruppen von Personen mit ähnlichen Lesevorzügen erstellen lassen. Je nachdem, welche Bücher die Personen bereits gelesen haben, kann somit ein \glqq{}Pool\grqq{} aus Büchern erstellt werden. In unserer Seminarfacharbeit wird vor allem auf die Erstellung des Programms, das Sammeln von Daten dafür und die Anwendung dessen eingegangen.

\section{Mathematische Prinzipien}

\section{Anwendung dieser Prinzipien}

\section{Erklärung / Vorstellung des Programms}

\subsection{Funktionsweise des abstrakten Teils}

\subsection{Erweiterung auf unsere Anwendung / Anpassung des Programms}

\subsection{Anwendung des Programms}
\end{multicols}
\end{document}

% useful things:
% --- TABLES ---
% \begin{table}[H]
% \caption{Example table}
% \centering
% \begin{tabular}{llr}
% \toprule
% \multicolumn{2}{c}{Name} \\
% \cmidrule(r){1-2}
% First name & Last Name & Grade \\
% \midrule
% John & Doe & $7.5$ \\
% Richard & Miles & $2$ \\
% \bottomrule
% \end{tabular}
% \end{table}